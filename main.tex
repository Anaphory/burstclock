\documentclass[a4paper,12pt]{scrartcl}

% Title page
\usepackage{authblk}

% General style
\usepackage{fontspec}
\setmainfont[BoldFont = GenBasB.ttf]{Gentium Plus}
\setsansfont{DejaVu Sans}
\usepackage{newunicodechar}
\newunicodechar{→}{\fontspec{Gentium Plus}→}

% Formulas
\usepackage{amsmath}

% Internal references
\usepackage[titletoc,title]{appendix}
\usepackage{hyperref}
\usepackage[capitalize]{cleveref}
\Crefname{appsec}{Appendix}{Appendices}
\crefname{appsec}{appendix}{appendices}

% Draft style
\usepackage{setspace}
\setstretch{1.5}
\usepackage[top=2.5cm, bottom=2.5cm, left=2.2cm, right=3.5cm]{geometry}
\usepackage{lineno} 

% Figures
\usepackage{subcaption}
\usepackage[font=small,labelfont=it]{caption}
\usepackage{graphicx}

% Bibliography
\usepackage[backend=biber,
            bibstyle=biblatex-sp-unified,
            citestyle=sp-authoryear-comp,
            maxcitenames=2,url=false,
            maxbibnames=99]{biblatex}
\renewcommand*{\bibfont}{\small}
\addbibresource{library.bib}

\renewbibmacro*{doi+eprint+url}{%
  \printfield{doi}%
  \newunit\newblock%
  \iftoggle{bbx:eprint}{%
    \usebibmacro{eprint}%
  }{}%
  \newunit\newblock%
  \iffieldundef{doi}{%
    \usebibmacro{url+urldate}}%
  {}%
}


% Code inclusion with syntax highlighting
\usepackage{minted}
\setminted{fontsize=\tiny,baselinestretch=0.9}

\title{Burst Clock}
\date{\today}
\author[1]{Nico Neureiter}
\author[1]{Gereon A. Kaiping}
\affil[1]{Geographic Information Science Center, Universität Zürich, CH}

\begin{document}
\maketitle
\begin{abstract}
  Due to increased rates of evolution at cladogenenesis, potentially arising
  from linguistic founder effects or the exaggeration of language differences to
  mark social boundaries, lanugages tend to evolve in bursts.
  \Textcite{atkinson2008languages} showed that a significant proportion of up to
  a third of lexical disparity arises from such punctuated events at splitting
  events. Similar bursts of evolution also at initial niche settlement in
  biology. Given the prevalence with which phylogenetic models in linguistics
  are used to date language families, it is surprising that this result has not
  found its way back into the clock models used in phylogenetic inferences,
  which tend to focus instead on per-branch variation \parencite{}. Here, we
  develop a simple extension to the Strict Clock model \parencite{} in BEAST2
  \parencite{} to account for bursts at language splits. We fit it to four major
  language families of the world –~Austronesian, Bantu, Indo-European, and Sino-Tibetan~– and find that the Burst Clock fits the data better than either
  a Strict Clock or an Uncorrelated Relaxed Clock, and on average a binary split
  is associated with XXX lexical changes in the basic vocabulary.
\end{abstract}

\section{Introduction}
Linguistic differences, in particular those apparent to speakers, are often
exaggerated among different social groups to mark ethnic boundaries. This
process, dubbed schismogenesis by Bateson (1935) and esoterogeny by Thurston
(1987), is thought to drive language disparity in as wide a range of locations
as Vanuatu \parencite{}, South America \parencite{} and …. Here, languages
function to identify in-groups, and as such they mark social boundaries (Labov
1963) and define potential marriage partners in exogamous societies
\parencite{}. These processes presumably happen in particular at times where
languages split, driving the schism and in turn being driven by the exaggerated
observation of differences. Lineages with more language diversification will
thus be more different from the ancestral languages than stable languages.
Quantitative evidence for this effect was supplied by
\textcite{atkinson2008languages}. Using lexical cognate data, they studied the
signal of punctuation in the Austronesian, Bantu, and Indo-European tree. The
lexicon is one of the domain of language most apparent to speakers and
accessible to conscious manipulation \textcite{}.
\Citeauthor{atkinson2008languages} found that between 9.5\% and 33\% of lexical
differences are due to punctuated events.

% If we want to talk about biological evolution: The relevant article cited in
% atkinson2008languages is N. Eldredge, S. J. Gould, in Models in Paleobiology,
% T. J. M. Schopf, Ed. (Freeman, San Francisco, 1972), pp. 82–115.

\Textcite{gray2013three} raise the question whether this holds equally for
different linguistic domains. They argue that “[t]o the extent that closely
related speech communities differ more in accent than they do in vocabulary, and
more in vocabulary than in language structure, it might be predicted that the
schismogenesis effect would be most pronounced in phonetics and least in
structural features of language.” In order to explicitly test this prediction,
an explicit model of punctuated evolution would be necessary.

The big driver of phylogenetic methods in linguistics has been their ability to
date language histories based on a relatively small number of broad calibration
points \parencite{}. Such models would also profit greatly from explicit
modelling of the punctuated evolution.

\paragraph{The structure of the paper} is as follows. In section
\cref{sec:description}, we describe the Burst Clock model, a generalization of
the Strict Clock. We provide and describe a BEAST2 \parencite{beast2} package
implementing this clock. We then describe the Bayesian phylogenetic inference on
lexical data of Austronesian, Bantu, Indo-European, and Sino-Tibetan data in
\cref{s:lexical}. In \cref{s:domains} we expand the Indo-European model to
incorporate phonological and grammatical features. We present our results in
\cref{s:results}. In \cref{s:discussion}, we discuss the results. In
\cref{s:conclusions} we formulate our conclusions.

Data and methods used to produce the results and figures shown in this paper are
available as online supplementary material and through
\url{https://osf.io/??????????????????}.

\section{Introducing the Burst Clock}
\label{sec:description}
In a strict clock \parencite{}, a constant rate $c$ maps branch lengths $l_i$,
measured in time (years, millennia, megayears) to expected numbers of changes
along a branch
\begin{align}
  e_i = c \cdot l_i
  \label{eq:strict}
\end{align}
In the most mathematically simple case,
every observed split (i.\,e. any split where descendants on both sides survived
to the be observed later) contributes a constant amount of expected changes to
the evolutionary history of the descendant languages, and there are no
unobserved splits in the tree.
That is, in addition the parameter $c$, this clock model has another
non-negative parameter $b$ and the expected number of changes is 
\begin{align}
  e_i = c \cdot l_i + b
  \label{eq:simple-burst}
\end{align}
For $b=0$, this models reduces to the strict clock.

Like the strict can be relaxed, eg. by drawing per-branch clock rates $c_i$ from
a lognormal distribution \parencite{}, so can the parameter $b$ be extended to
per-node burst values $b_i$. For $c=0$, this gives rise to non-clock models, and
with an appropriate model of the $b_i$ can represent various non-clock priors,
such as the compound Dirichlet prior (Rannala et al. 2012).
% https://academic.oup.com/sysbio/article/61/5/779/1735441

In addition, the unobserved splits, where a language splits in two, but one of
the two descendant branches goes extinct before being observed, can contribute
to language change. In a birth-death tree model \parencite{}, the rate at which
languages split (birth rate $\lambda$) is an explicit parameter of the model, so
we can estimate the expected number of splits on a branch as $\lambda \cdot
l_i$, giving rise to the expected number of changes
\begin{align}
  e_i = (c + \lambda b) \cdot l_i + b
  \label{eq:reparam-burst}
\end{align}
This is formally the same model as in \cref{eq:simple-burst}, but parameterized
in a way that takes the underlying tree prior into account.

Most general case we can maybe construct:
\begin{align}
  e_i = c_i l_i + \left[Draw \left( [\text{draw number of split events on }b\text{ with rate } \lambda] + 1 \right) b\text{-values}\right]
  \label{eq:reparam-burst}
\end{align}
($c_i$ drawn according to a non-strict clock, $b$s drawn according to a non-clock prior, eg. Dirichlet Gamma)
\end{document}
