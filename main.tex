\documentclass[a4paper,12pt]{scrartcl}

% Title page
\usepackage{authblk}

% General style
\usepackage{fontspec}
\setmainfont[BoldFont = GenBasB.ttf]{Gentium Plus}
\setsansfont{DejaVu Sans}
\usepackage{newunicodechar}
\newunicodechar{→}{\fontspec{Gentium Plus}→}

% Formulas
\usepackage{amsmath}
\usepackage{amssymb}

% Internal references
\usepackage[titletoc,title]{appendix}
\usepackage{hyperref}
\usepackage[capitalize]{cleveref}
\Crefname{appsec}{Appendix}{Appendices}
\crefname{appsec}{appendix}{appendices}

% Draft style
\usepackage{setspace}
\setstretch{1.5}
\usepackage[top=2.5cm, bottom=2.5cm, left=2.2cm, right=3.5cm]{geometry}
\usepackage{lineno} 

% Figures
\usepackage{subcaption}
\usepackage[font=small,labelfont=it]{caption}
\usepackage{graphicx}

% Bibliography
\usepackage[backend=biber,
            bibstyle=biblatex-sp-unified,
            citestyle=sp-authoryear-comp,
            maxcitenames=2,url=false,
            maxbibnames=99]{biblatex}
\renewcommand*{\bibfont}{\small}
\addbibresource{library.bib}

\renewbibmacro*{doi+eprint+url}{%
  \printfield{doi}%
  \newunit\newblock%
  \iftoggle{bbx:eprint}{%
    \usebibmacro{eprint}%
  }{}%
  \newunit\newblock%
  \iffieldundef{doi}{%
    \usebibmacro{url+urldate}}%
  {}%
}

\newcommand{\glot}[2]{#1 [\texttt{\hyperlink{https://glottolog.org/resource/languoid/id/#2}{#2}}]}

% Code inclusion with syntax highlighting
\usepackage{minted}
\setminted{fontsize=\tiny,baselinestretch=0.9}

\title{Burst Clock}
\date{\today}
\author[1]{Nico Neureiter}
\author[1]{Gereon A. Kaiping}
\affil[1]{Geographic Information Science Center, Universität Zürich, CH}

\begin{document}
\maketitle
\begin{abstract}
  Due to increased rates of evolution at cladogenenesis, potentially arising
  from linguistic founder effects or the exaggeration of language differences to
  mark social boundaries, lanugages tend to evolve in bursts.
  \Textcite{atkinson2008languages} showed that a significant proportion of up to
  a third of lexical disparity arises from such punctuated events at splitting
  events. Similar bursts of evolution also at initial niche settlement in
  biology. Given the prevalence with which phylogenetic models in linguistics
  are used to date language families, it is surprising that this result has not
  found its way back into the clock models used in phylogenetic inferences,
  which tend to focus instead on per-branch variation \parencite{}. Here, we
  develop a simple extension to the Strict Clock model \parencite{} in BEAST2
  \parencite{} to account for bursts at language splits. We fit it to four major
  language families of the world –~Austronesian, Bantu, Indo-European, and Sino-Tibetan~– and find that the Burst Clock fits the data better than either
  a Strict Clock or an Uncorrelated Relaxed Clock, and on average a binary split
  is associated with XXX lexical changes in the basic vocabulary.
\end{abstract}

\section{Introduction}
Linguistic differences, in particular those apparent to speakers, are often
exaggerated among different social groups to mark ethnic boundaries. This
process, dubbed schismogenesis by Bateson (1935) and esoterogeny by Thurston
(1987), is thought to drive language disparity in as wide a range of locations
as Vanuatu \parencite{}, South America \parencite{} and …. Here, languages
function to identify in-groups, and as such they mark social boundaries (Labov
1963) and define potential marriage partners in exogamous societies
\parencite{}. These processes presumably happen in particular at times where
languages split, driving the schism and in turn being driven by the exaggerated
observation of differences. Lineages with more language diversification will
thus be more different from the ancestral languages than stable languages.
Quantitative evidence for this effect was supplied by
\textcite{atkinson2008languages}. Using lexical cognate data, they studied the
signal of punctuation in the Austronesian, Bantu, and Indo-European tree. The
lexicon is one of the domain of language most apparent to speakers and
accessible to conscious manipulation \textcite{}.
\Citeauthor{atkinson2008languages} found that between 9.5\% and 33\% of lexical
differences are due to punctuated events.

% If we want to talk about biological evolution: The relevant article cited in
% atkinson2008languages is N. Eldredge, S. J. Gould, in Models in Paleobiology,
% T. J. M. Schopf, Ed. (Freeman, San Francisco, 1972), pp. 82–115.

% There is also a paper by Søren Wichman and Taraka Rama on ‘Bayesian Dating’ or something, I think it basically does linear interpolation, so it's also case in point?

\Textcite{gray2013three} raise the question whether this holds equally for
different linguistic domains. They argue that “[t]o the extent that closely
related speech communities differ more in accent than they do in vocabulary, and
more in vocabulary than in language structure, it might be predicted that the
schismogenesis effect would be most pronounced in phonetics and least in
structural features of language.” In order to explicitly test this prediction,
an explicit model of punctuated evolution would be necessary.

The big driver of phylogenetic methods in linguistics has been their ability to
date language histories based on a relatively small number of broad calibration
points \parencite{}. Such models would also profit greatly from explicit
modelling of the punctuated evolution.

\paragraph{The structure of the paper} is as follows. In section
\cref{sec:description}, we describe the Burst Clock model, a generalization of
the Strict Clock. We provide and describe a BEAST2 \parencite{beast2} package
implementing this clock. We then describe the Bayesian phylogenetic inference on
lexical data of Austronesian, Bantu, Indo-European, and Sino-Tibetan data in
\cref{s:lexical}. In \cref{s:domains} we expand the Indo-European model to
incorporate phonological and grammatical features. We present our results in
\cref{s:results}. In \cref{s:discussion}, we discuss the results. In
\cref{s:conclusions} we formulate our conclusions.

Data and methods used to produce the results and figures shown in this paper are
available as online supplementary material and through
\url{https://osf.io/??????????????????}.

\section{Introducing the Burst Clock}
\label{sec:description}
In a strict clock \parencite{}, a constant rate $c$ maps branch lengths $l_i$,
measured in time (years, millennia, megayears) to expected numbers of changes
along a branch
\begin{align}
  e_i = c \cdot l_i
  \label{eq:strict}
\end{align}
In the most mathematically simple case,
every observed split (i.\,e. any split where descendants on both sides survived
to the be observed later) contributes a constant amount of expected changes to
the evolutionary history of the descendant languages, and there are no
unobserved splits in the tree.
That is, in addition the parameter $c$, this clock model has another
non-negative parameter $b$ and the expected number of changes is 
\begin{align}
  e_i = c \cdot l_i + b
  \label{eq:simple-burst}
\end{align}
For $b=0$, this models reduces to the strict clock.

Like the strict clock can be relaxed, eg. by drawing per-branch clock rates $c_i$ from
a lognormal distribution \parencite{}, so can the parameter $b$ be extended to
per-node burst values $b_i$. For $c=0$, this gives rise to non-clock models, and
with an appropriate model of the $b_i$ can represent various non-clock priors,
such as the compound Dirichlet prior (Rannala et al. 2012).
% https://academic.oup.com/sysbio/article/61/5/779/1735441

In addition, the unobserved splits, where a language splits in two, but one of
the two descendant branches goes extinct before being observed, can contribute
to language change in this manner. In a birth-death tree model \parencite{}, the rate at which
languages split (birth rate $\lambda$) is a parameter of the model, either given explicitly or in terms of net diversification rate and turnover. Therefore
we can estimate the expected number of splits along a branch as $\lambda \cdot
l_i$. If turnover is low and many tips are sampled, this over-estimates the number of branches, because it does not take into account that the majority of branches are actually observed. For a sampling proportion $q$ and $l$ observed leaves, the number of net splits will be
$\frac{l}{q} - 1$
while the observed number of splits will be $l - 1$. This means that the proportion of unobserved splits is, on average,
\begin{align}
    \rho_u &= \frac{(\frac{l}{q} - 1) - (l - 1)}{\frac{l}{q} -1} \\
    & = \frac{l - q - ql + q}{l - q} \\
    & = \frac{l - ql}{l - q} \\
    & \approx 1 - q
\end{align}
giving rise to the expected number of changes
\begin{align}
  e_i = (c + \lambda (1-q) b) \cdot l_i + b
  \label{eq:reparam-burst}
\end{align}
This is formally the same model as in \cref{eq:simple-burst}: The expected number of changes is linear (with a non-zero intercept) in the branch length. It is, however, parameterized
in a way that takes the underlying tree prior into account,
which should lead to better mixing and better interpretability.

\section{Bayesian phylogenetic inference}

In order to assess whether the burst clock is a useful model for Bayesian phylogenetics, we reproduce four studies on
Bayesian phylogenetics on well-studied language phylogenies: Austronesian \parencite{}, Bantu \parencite{}, Indo-European
\parencite{} and Sino-Tibetan \parencite{}. We first use a state-of-the-art inference model with a non-burst clock, as a
baseline comparable with published studies, before adding a burst clock to the inference procedure. We assess the posterior
probability distribution of the added parameter $b$, and compare model fit using Stepping Stone analysis implemented in BEAST2's
\texttt{modelselection} package \parencite{}.

\subsection{Data}
The data for the phylogenetic analysis is composed of lexical items in comparable concepts (“word list” or “concept list”) annotated
for cognacy, i.\,e. presumed inheritance from a shared root. Dated trees with meaningful branch lengths and clock data also require calibration points.
These are specific to the language families and will also be discussed in the following.

\subsubsection{Austronesian}

            \newcommand{\countlects}{315}
            \newcommand{\countconcepts}{177}
            \newcommand{\ncharacters}{10627}
            

The Austronesian language family is …
Due to the wide spread, these languages have been regularly studied using phylogenetics, eg. by
\textcite{gray2009language,greenhill2017evolutionary,greenhill2018population}.
\paragraph{Lexical Data}
Of these articles using lexical data to infer Austronesian language phylogenies, only the second one
% gray2009language: “The data used in this study were extracted from the Austronesian Basic Vocabulary Database (“ABVD”) project (Greenhill et al. 2008). We selected the 400 languages/dialects with the most available data, and excluded known creoles and languages with large amounts of admixture (Table S1). We included two non-Austronesian languages as outgroups to “root” the trees: an archaic variant of the Sino-Tibetan language Chinese that was spoken between 2,300 and 2,900 years ago, and theTai-Kadai language Buyang. We extracted the cognate set data for these 400 languages across all 210 wordlist items. These data were translated into a binary matrix representing the presence or absence of each cognate set in each language.”
% greenhill20818population: “For the Austronesian languages we used the Austronesian Basic Vocabulary Database (ABVD, Greenhill et al., 2008) which contains wordlists for 210 semantic categories from 1,278 languages.”, not even mentioning data coding, no supplement
% greenhill2017evolutionary: “We used the ABVD (17) to find lexical data for the languages in the PIMdb. We identified 81 languages in both the PIMdb and the ABVD. [...] All lexical and cognate information is available online at https://abvd.shh.mpg.de/austronesian, and the cognate file is available in SI Materials and Methods and Dataset S4. are published with the acutal data in supplement.
\parencite{greenhill2018population} makes the associated lexical data available, instead of only describing the data gathering procedure.
Hovever, in that case the dataset is reduced to the overlap with a grammatical dataset.
All three papers use the Austronesian Basic Vocabulary Database for their lexical data, but the official version of that database
(https://abvd.shh.mpg.de/austronesian/) is published without a persistent identifier, not freely usable, and not available in a re-usable format, so we instead make use of the ABVD
data in the standardized CLDF format \parencite{cldf} curated on \url{https://github.com/lexibank/abvd}, which is published under a
permissive Creative Commons (CC-BY-4.0) license. We used the most recent version, commit \texttt{17a4b30922f9d1010667ae8974f814c624e3e9a4}
(2020-07-10). We filtered filtered to the Austronesian doculects\footnote{The database contains several instances where different
sources for the same variety (judging by name and Glottocode) give rise to different ‘language’ objects in the dataset, so the term
‘doculect’ appears more appropriate than ‘language’ or ‘dialect’ in discussing the composition of the dataset, even though some languages may be aggregated from more than one source.} that have a coverage
of at least 85\% of all concepts (189 or more concepts). This results in \countlects{} doculects and \countconcepts{} concepts being included in the data.

\paragraph{Calibrations}
\newcommand{\abvd}[2]{#1 [\hyperlink{https://abvd.shh.mpg.de/austronesian/language.php?id=#2}{#2}]}
% gray2009language: clades not explicitly tied to calibrations, only description needing interpretation in Table S3
% greenhill2018population: No calibrations
% greenhill2017: To calibrate these clocks, we incorporated historical evidence of language divergence times as described by Gray et al. (50). We implemented five calibrations on the tree using normally distributed priors on the node heights. These calibrations were the following: (i) Proto-Oceanic (mean of 3,300 y, SD = 100 y), (ii) Proto-Central Pacific (mean of 3,000 y, SD = 100), (iii) Proto-Malayo-Polynesian (mean of 4,000 y, SD = 250), (iv) Proto-Micronesian (mean of 2,000 y, SD = 100), and (v) Proto-Austronesian (mean of 5,200 y, SD = 300).
Following \textcite{greenhill2017evolutionary}, we calibrate five nodes using normally
distributed priors. We supplement this with age constraints taken from \textcite[Table S3]{gray2009language}. There, we interpret the ranges given there as $2\sigma$ intervals of a normal distribution. This leads to the following node calibrations, sorted by their mean. We do not constrain the languages with these calibrations to be clades: a given calibrated MRCA node may be ancestral to more languages than explicitly listed.
\begin{itemize}
\item ‘Proto-Austronesian (mean of 5,200 y, SD = 300)’, i.\,e. on the root of our tree;
\item ‘Proto-Malayo-Polynesian (mean of 4,000 y, SD = 250)’, i.\,e. on the most recent common ancestor (MRCA) of all languages inside the Glottolog family \glot{Malayo-Polynesian}{mala1545};
\item ‘Proto-Oceanic (mean of 3,300 y, SD = 100 y)’, i.\,e. on the MRCA of \glot{Oceanic}{ocea1241};
\item \glot{Central Pacific}{cent2060} $\mu=3000, \sigma=100$;
\item \glot{Micronesian}{micr1243} $\mu=2000, \sigma=100$;
\item \glot{Malayo-Chamic}{nort3170}: $\mu=1500, \sigma=250$.
\item \glot{Tuvalu and Tokelau}{elli1239}: $\mu=1500, \sigma=250$;
\item \glot{East Polynesian}{east2449}: $\mu=1475, \sigma=167.5$; 
\item \glot{Malagasic}{mala1537}: $\mu=1200, \sigma=50$;
\item \glot{Javanese}{java1253}: $\mu=1200, \sigma=50$;
\item \glot{Chamic}{cham1330}: $\mu=1150, \sigma=175$;
\end{itemize}
The only \glot{Reefs-Santa Cruz language}{reef1242} in our Austronesian lexical dataset is \abvd{Äiwoo}{501}, so a calibration of the most recent ancestor of all Reefs-Santa Cruz languages is not possible.

In terms of tip calibrations given by \textcite{gray2009language}, table row on \abvd{Old Javanese}{290} cites \textcite{zoetmulder1982old} for a date between 700 and 1200 BP ($\mu=950, \sigma=125$).
All their other tip calibrations either do not apply to our filtered data set or are dubious:
For \abvd{Favorlang}{831}, some of the derives from Dutch sources from the mid-1600s, others from wordlists collected around 1900. Such a mix of data makes it unsuitable for dating, so we excluded \abvd{Favorlang}{831} from the lexical data.

\subsubsection{Bantu}
The Bantu languages are a part of the Atlantic-Congo languagage family. They
have been studied in lexical phylogenetics by
\textcite{grollemund2015bantu,greenhill2018population,currie2013cultural}.
\paragraph{Lexical Data}
% grollemund2015bantu: For each of the n=100 lexical items (meanings), we have used the comparative method wherever possible to identify cognate sets(words with the same meaning that derive from a common ancestor). Where it was not possible to establish strict correspondences for every word, we based our cognacy judgment on the principle of resemblance. This work was conducted by R.G. as part of her PhD and postdoctoral work on the Bantu languages (66). We identified 3,859 cognate sets across then=100 meanings.These were coded as binary characters for purposes of phylogenetic analysis. In practice, expert opinion on cognate classifications can differ (this difference also occurs in the alignment of gene sequence data where it is necessary to identify homologous genes), so we have conducted a series of analyses to check that our principal results are robust to variation in the data. We created subsampled datasets, with each one consisting of 50meanings randomly sampled without replacement from the data. These datasets were then converted to a binary matrix from which we inferred the tree. We repeated this procedure 100 times. We found that in 98% of these random samples based on just half the data, the tree we inferred showed the ladderized or pectinate backbone that we reported for the full-dataset tree in Fig. 1. This result ensured that the signal for the tree we use to infer the Bantu migration route was robust to variation in the data
% greenhill2018population: “Basic vocabulary for 100 words from 409 Bantu languages were provided by Grollemund et al. (2015) in a phylogenetic dataset that records a single variant per semantic category for each language.
% currie2013cultural: “Here, we use linguistic data from 542 spoken varieties of Bantu [16]. The names and alpha-numeric codes of all these languages are listed in the electronic supplementary material, table S1. Linguistic data in the form of different lexical items (‘words’) were taken from Bastin et al. [16]. These data code whether these basic vocabulary words from different languages can be considered cognate (i.e. they share a common origin). To facilitate phylogenetic analyses, these data were recoded into binary cognate sets reflecting the presence or the absence of each cognate in each language.
The supplementary data from the 2015 paper is available from
\url{http://www.evolution.reading.ac.uk/DataSets.html}. A CLDF version of this dataset by Robert Forkel, Tiago Tresoldi, Mark Pagel, and Rebecca Grollemund, published under the CC-BY-NC 4.0 license, can be found on \url{https://github.com/lexibank/grollemundbantu}. We use version v1.0rc6 of the dataset.

\paragraph{Calibrations}
% greenhill2018population: No calibrations
% grollemund2015bantu: We used archaeological data to propose date ranges, and in one case a fixed date, for four nodes of our tree (labeled a–d in Fig. 1). The four calibrations are as follows: (a) 5,000 B.P. or older for Bantoid, non-Bantu (58); (b) 4,000–5,000 B.P. for Narrow Bantu (13, 14, 16, 44, 59, 60); (c) 3,000–3,500 B.P. for the Mbam-Bubi ancestor (61); and (d) 2,500 B.P. for Eastern Bantu (62). We used a uniform prior in our Bayesian tree inference for all calibration ranges.
We take the calibration ranges from \textcite{grollemund2015bantu}, but generally use normal distributions instead of uniform distributions. Again we do not enforce monophyly, which is particularly important for this language family, where two of the calibrated language groups (Bantoid and Mbam-Bubi) are paraphyletic in Glottolog.
\begin{itemize}
\item A uniform prior above 5000~yBP (with upper limit 20000~yBP) for the MRCA of the non-Bantu Bantoid languages (aghemgrassfields, njengrassfields, mbulajarawan, bamungrassfields, fefegrassfields, okugrassfields, dugurijarawan, moghamograssfields, bwazzajarawan, komgrassfields, bilejarawan, kulungjarawan, mungakagrassfields, zaambojarawan, tivtivoid);
\item For the languages in \glot{Narrow Bantu}{narr1281}, a normal distribution with $\mu=4500, \sigma=250$;
\item For the MRCA of the \glot{Mbam languages}{mbam1252} and \glot{Bubi}{bubi1250}, $\mu=3250, \sigma=125$;
\item For \glot{East Bantu}{east2731}, $\mu=2500, \sigma=50$.
\end{itemize}

\subsubsection{Indo-European}
For obvious reasons, Indo-European has been extensively studied using lexical Bayesian phylogenetics since the beginning
of that field
\parencite{bouckaert2012mapping,chang-ie,gray2003language,holm2017steppe,rama2018three,willems2016using}

\paragraph{Lexical Data}
Datasets generally go back to \textcite{dyen1992comparative} and its derivatives such as IELex \parencite{ielex},
but have been improved
step by step. The most recent study correcting
the quality of cognate codes is by \textcite{chang-ie}. Their different datasets are explicitly compared by \textcite{rama2018three}, who finds that the
‘\textsc{medium}’ dataset fits the expected node ages best given a uniform tree prior. \Citeauthor{rama2018three} also argues qualitatively in favour of
that dataset over the alternatives considered, so we base our analysis on that dataset, available from
\url{https://github.com/PhyloStar/ie-phylo-exps/blob/a63c0b52f7772adc2341572932a74b23d92570df/medium1.nex}.

\paragraph{Calibrations}
Following the evidence presented by \textcite[Tables 7 and 12]{chang-ie}, we implement the following node and tip calibrations.
\begin{itemize}
    \item 
\end{itemize}

\subsubsection{Sino-Tibetan}
The Sino-Tibetan language family has been recently studied phylogenetically by two independent groups
\parencite{sagart2019dated,zhang2019phylogenetic}.
\paragraph{Lexical Data}
The lexical data is available as supplementary material for each of these two
articles. Both data sets are ultimately derived from the STEDT dataset
\parencite{}. \Citeauthor{sagart2019dated} are very explicit concerning their
data selection and cognate coding method, while
\Citeauthor{zhang2019phylogenetic} retain a larger language sample. While
smaller, the \textcite{sagart2019dated} dataset seems thus much more carefully
curated, and is thus preferable for our analysis.

\paragraph{Calibrations}
We follow \textcite{sagart2019dated} in the tip calibrations, ‘Old Chinese, [2,800 to 2,300] yBP, in a uniform prior; Old Burmese, 800 yBP; Old Tibetan, 1,200 yBP; and Tangut, 900 yBP’.
% sagart2019dated: We specified calibrations as follows: Old Chinese, [2,800 to 2,300] yBP, in a uniform prior; Old Burmese, 800 yBP; Old Tibetan, 1,200 yBP; and Tangut, 900 yBP (the date range for Old Chinese corresponds to the period of the great Classical Chinese texts; the other dates correspond to the date of the earliest text rounded to the nearest century; see also SI Appendix, section 4).
% zhang2019: Supplementary table 2 (https://static-content.springer.com/esm/art%3A10.1038%2Fs41586-019-1153-z/MediaObjects/41586_2019_1153_MOESM1_ESM.pdf):
% Chinese Normally distributed with mean 2700.0 and 150.0 standard deviation The earliest known written records of the Chinese language dates from about 1250 BC.
% Old Chinese Normally distributed with mean 2500.0 and 100.0 standard deviation. Tips Only.
% Tibetan dialects Normally distributed with mean 1150.0 and 50.0 standard deviation Tibetan dialects were formed in the mid-13th century. Tibetan dialects exclude the classical Tibetan.
% Burmese Truncated between 400.0 and 1200.0 The language is not attested until early in the 12th century, when it begins to appear on tone inscriptions in the temples of Pagan.
% Pumi Normally distributed with mean 750.0 and 50.0 standard deviation Pumi people migarated into the regions of Ninglang, Lijiang, Weixi and Lanpin from 13th century.
% Yi Normally distributed with mean 1500.0 and 100.0 standard deviation The expansion of Yi people to northeast and south Yunnan, and northwest Guizhou dated to 3th century. About in the eighth and ninth century, a kindom called Nazhao was ruled by Yi (Lolo) speakers but the people also included Bai language.

\section{Methods and Models}
We infer a dated language phylogeny for each of the datasets described above.
\paragraph{Data coding and substitution model}
We use the same binary root-meaning presence/absence coding also used in the
cited phylogenetic studies. We account for the ascertainment bias that the data
only includes cognate sets that are attested in any contemporary language.
While the binary covarion model has been shown to outperform the binary CTMC
model on several occasions, its main purpose is to absorb varying substitution
rates along branches. Substitutions are highly informative for our burst clock,
so we model substitutions using a binary CTMC model and expect that the varying
rates of substitution along branches instead inform the branch lengths,
per-branch clock rates and the parameter $b$ of our burst clock.
Following \parencite{}, we assume that substitution rates are constant within a
meaning class, but that the substitution rates between different meaning classes
are distributed according to a Gamma distribution with mean 1 and shape 1.

\paragraph{Tree prior} We assume a fossilized birth/death tree prior, with a low
turnover of 20\% \parencite{} and a sampling proportion derived from the number
of contemporary tips in the dataset divided by the family's number of languages
according to Glottolog \parencite{glottolog}. We take the net diversification
rate to follow a lognormal prior with 95\% of the probability mass between “one
split per generation” and “one split per 1000 years”,
% ln 1/20 = -2.995732273553991 = x1
% ln 1/1000 = -6.907755278982137 = x2
% 95\% interval == 2 σ interval, so
% μ = (x1 + x2) / 2 = -4.951743776268064
% σ = (x1 - x2) / 4 = 0.9780057513570365
i.\,e. with mean $-4.951743776268064$ and standard deviation $0.9780057513570365$ in log space.

\paragraph{Clock model}
To account for rate variation between different parts of the phylogeny, we
assume a lognormal relaxed clock. This means that the branch rate $c_i$ on a
branch $i$ is drawn from a lognormal distribution with mean $1$ and log-stdev
$???$, independent of every other branch. In a secondary analysis step, we
assume a burst clock with split bursts $b$ distributed according to an
exponential distribution with mean $1$ (so favouring small bursts, but
permitting any number of changes per split) and brach rates $c_i$ following the
same lognormal relaxed clock as before.

\paragraph{Phylogenetic inference}
We infer the phylogenies using Markov chain Monte Carlo sampling, using the MCMC
software tool BEAST2 \parencite{beast2}. We run the Markov chain for 10'000'000
steps, sampling every 1'000th step and discarding the first 10\% as burn-in,
thus creating 900 sampled phylogenies, which we compare to the published
phylogenies. We use a stepping stone analysis for estimating marginal model
likelihoods using the modelselection package \parencite{modelselection} for
BEAST2 to compare the marginal likelihood of the models with and without bursts
at language splits.

The Python source code to generate the BEAST2 configuration files, as well as
those XML files themselves and the raw log file outputs of the BEAST runs are
available in the supplementary material.

\section{Results}

\printbibliography{}


\end{document}

% Local Variables:
% TeX-engine: luatex
% End:
